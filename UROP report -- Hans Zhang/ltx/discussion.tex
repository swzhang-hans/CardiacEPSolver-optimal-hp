The contour plots presented in this paper show the general trends. However, they may not accurately reflect singular cases. For example, Figure \ref{hp_unstr-tri} demonstrates an unexpected accuracy spike at specific polynomial orders, despite the overall expectation that higher polynomial orders should yield better accuracy, assuming other parameters remain constant. The CardiacEPSolver handles the cell model and the PDE separately using different numerical methods featuring different computation times and errors. Because of this, depending on whether the diffusion or the reaction process dominates the propagation, which is determined by the numerical values in the monodomain equation, the optimal $hp$ combination for a desired accuracy may vary. As indicated in Figure \ref{coeff}, error reduction tends to occur when the PDE is more diffusion-dominated. The reduction in runtime with nodal point distribution was anticipated, as nodal distribution uses fewer nodes compared to Gaussian distribution. However, this reduction comes at the cost of solution quality, particularly in terms of capturing the uniformity of the wavefront, as shown in Figures \ref{distri_str_origin_std} and \ref{distri_str_nodal_std}. While nodal distribution might offer a performance advantage, its impact on accuracy should be carefully considered, especially in cases where the precision of the geometry is important.
\par
Limitations in this study are recognised. Firstly, the reliability of the results is affected by the filtering and interpolation applied to the data. More refined simulations with finer incremental steps in $h$ could have reduced the need for interpolation, providing a clearer representation of trends. Moreover, alternative methods for error computation should be explored to minimise bias and improve the accuracy of general trend identification. One potential improvement could be utilising the built-in interpolation feature in Nektar++ to map the fine solution onto the coarse discretisation's solution field and compute the $L2$ error directly. Secondly, the unstructured triangular mesh used in this study still displayed some degree of orderliness due to the rectangular geometry of the domain. As a result, the findings may not fully generalise to truly unstructured mesh configurations. Future work should explore more complex geometries to assess the robustness of the methods. Thirdly, the code for nodal distribution is still under development and has not been rigorously tested for its functionality. Further validation is needed to ensure its accuracy and reliability. Additionally, nodal distribution should be tested under varying mathematical coefficients in the PDE to better understand its impact on the optimal $hp$ combination. For example, by using a more reaction-dominated PDE, the effects of nodal distribution on solving the cell model may become more pronounced. This would allow for a more comprehensive assessment of how the mathematical coefficients influence the performance of different $hp$ combinations.


