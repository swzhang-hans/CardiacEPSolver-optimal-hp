It is important to emphasise that the dicretisation is governed by a number of numerical parameters and the optimal choice is sensitive to the nature of the PDE. This paper only explores the tuning of element size ($h$) and polynomial order ($p$) for solving reaction-diffusion PDE at varying conductivities and with Gaussian and electrostatic quadrature point distributions. 
\par
Among the three types of mesh tested, the unstructured triangular mesh is likely of most interest to users, as it better captures complex geometries compared to structured meshes. The optimal $hp$ combination ultimately depends on the desired level of accuracy. In general, increasing either the number of elements or the polynomial order leads to longer computation times. However, increasing the polynomial order beyond 7, or increasing the number of elements beyond $1/h = 14$, offers diminishing returns in terms of accuracy improvement. Additionally, a minimum number of elements is sometimes required, regardless of the polynomial order, to achieve a given accuracy. The contour plots presented in this paper can be directly referenced to guide users in selecting the appropriate discretization based on the conductivity and their specific accuracy and runtime requirements. For problems with lower conductivity—where the process is more diffusion-dominated—users should consider employing a finer mesh with a higher polynomial order to maintain accuracy, as indicated by the trends in Figure \ref{coeff}. When selecting a quadrature point distribution for the cell model, using electrostatic (nodal) distribution can significantly reduce runtime, but it may lead to reduced solution quality, particularly in capturing the wavefront. As the nodal distribution code in Nektar++ 5.6.0 has not been extensively tested, its use is not recommended at this time.
\par
Future studies should aim to explore the influence of other parameters on runtime and accuracy to further clarify and quantify the relationship between the optimal discretisation choices and the mathematical coefficients of the PDE. 