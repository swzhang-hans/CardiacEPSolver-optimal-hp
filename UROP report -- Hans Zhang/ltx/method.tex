\subsection{Problem setup}
The problem considered in this paper is the monodomain equation (Equation \ref{monodomain} \cite{RefWorks:clayton2011models}) on a rectangular domain $[0,10][0,1]$, where $\sigma$ and $C_m$ represent the coefficient of (isotropic) conductivity and the coefficient of cell membrane capacitance respectively. These two coefficients govern the diffusivity of the solution. 
\begin{equation}
    \chi [C_m \frac{\partial V_m}{\partial t} + J_{ion}] = \nabla \cdot (\sigma \nabla V_m)
    \label{monodomain}
\end{equation}
For the study of quadrature point, the domain was enlarged to $[0,10][0,5]$. The reason for doing so is explained in Section 2.2. The example problem was solved using CardiacEPSolver in Nektar++ 5.6.0. 
\par
The monodomain equation models the propagation of action potential (AP) generated by cardiac cells in human heart tissue. Such propagation is governed by both the spatial diffusion of action potential along the tissue and the physical-chemical reactions within the cells. The CardiacEPSolver supports a number of cell models. In this study, the Courtemanche model \cite{RefWorks:courtemanche1998ionic} (called CourtemancheRamirezNattel98 in the software) was used.  The solver handles the cell model and the PDE separately using different numerical methods for each. In this paper, we are interested in the choice of quadrature point distribution used in the cell model. To enable nodal distribution, users need to manually modify the C++ source code. The wave propagation was excited by applying a uniform stimulus to a region that covers the left boundary. It is trivial to see that the solution should be constant in the y-direction at any time step, resulting in a uniform wavefront. Simulations were conducted on three types of meshes: structured triangular mesh, unstructured triangular mesh and structured quadrilateral mesh. The meshes were generated using Gmsh 4.8.4.
\subsection{Experimental methods}
The element size ($h$) was quantified as the area of the domain divided by the number of elements generated. The ratio between the number of vertices on the x-axis and that on the y-axis was kept the same whenever possible to eliminate the influence of mesh shape on accuracy and runtime, particularly to avoid mesh distortion. The polynomial order studied ranged from 2 to 15, as this range is commonly considered high order in spectral/hp element method \cite{RefWorks:cantwell2011efficiently:}. To identify the optimal choice, simulations were run on all $hp$ combinations and their runtime and solution accuracy were compared. The same experiment was repeated with different values of conductivity $\sigma$ and membrane capacitance $C_m$ to observe how they affect the optimal $hp$ combinations. The conduction speed of the wavefront was used as a metric for solution accuracy. The wavefront was defined as the set of points with the highest voltages at each time step along the horizontal lines that these points lie on. The speed was calculated by measuring the time taken for the wavefront to travel between two fixed history points - a built-in feature supported by the solver. Since the problem does not have an analytical solution, a much finer mesh (with a $h$ that is 7 times smaller than the $h$ of the finest mesh used in other simulations) with a polynomial order of 18 was used to compute an approximated "true" solution. This fine solution was then used to compute the error in conduction speed. The runtime for each simulation was recorded using the solver's built-in function. 
\par
As mentioned earlier, the wavefront should be a straight line parallel to the y-axis. The uniformity (or variation) of the wavefront, calculated for both distributions, was used as a measure of solution quality. The uniformity was quantified as the standard deviation of the x-coordinates of the points on the wavefront at each time step. The standard deviation was computed for all time steps and the mean was taken for comparison. Additionally, the width of the domain was increased from 1 to 5 to increase the sample size to highlight the impact of point distribution on the variation of the wavefront. 

\subsection{Test system}
The computation time is highly dependent on the hardware used. In this paper, the study on the optimal $hp$ combination was conducted on a 64-bit system with dual 20-core 2.20 GHz Intel Xeon E5-2698 v4 processors and 251 GB of memory. The study on quadrature point choice was conducted on a 64-bit system with a 1.30GHz Intel Core i7-1065G7 processor and 16GB system memory. All visualisations and post-processing of simulation results were carried out using ParaView 5.11.0. The simulations used the implicit-explicit method (called IMEXdirk\_3\_4\_3 in the solver).